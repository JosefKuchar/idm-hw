\section*{Příklad 5}
Login studentů FITu je sled znaků, např. xabcde00, první znak je vždy písmeno x, potom následuje
pět písmen a poslední dva znaky jsou číslice. M je množina všech přípustných loginů. Login Li je
v relaci R s loginem Lj , právě když mají shodné alespoň první tři znaky. Zjistěte, zda R je relace
ekvivalence nebo uspořádání (případně ani jedno) na množině M. \textbf{Svoje tvrzení zdůvodněte.}\\\\
Reflexivnost relace $R$

Libovolný login $L$ má sám se sebou shodné všechny znaky

$\implies$ má shodné první tři znaky

$\implies$ relace $R$ je reflexivní \\
Symetrie relace $R$

Když máme login $Li$, který je v relaci s loginem $Lj$, tak mají shodné alespoň první 3 znaky a to platí i naopak $\implies$ Relace $R$ je symetrická \\
Tranzitivnost relace $R$

Když máme login $Li$, který je v relaci s loginem $Lj$ a zároveň pokud je login $Lj$ v relaci s $Lk$, tak $Li$ a $Lk$ mají shodné alespoň první 3 znaky $\implies$ Relace $R$ je tranzitivní \\\\\\
Relace $R$ je reflexivní, symetrická, tranzitivní $\implies$ Relace $R$ je relace ekvivalence
